We present two versions of  the  constructive (or intuionistic) modal logic S4, which we write as CS4. 
We start by recalling the basic sequent calculus for intuitionistic propositional logic. 

Each of the logics presented in this section is an extension of the
single-conclusion formalization of Gentzen's intuitionistic sequent
calculus LJ.  The syntax of formulas for LJ is defined by the
following grammar:
\begin{center}
    \begin{math}
        \begin{array}{lllllllll}
            A & ::= & p \mid \perp \mid A \land A \mid A \lor A \mid A \to B
        \end{array}
    \end{math}
\end{center}
The formula $p$ is taken from a set of countably many  propositional atoms. The constant $\top$ could be added, but it is the negation of the the falsum constant $\bot$.
%We begin with an initial set of inference rules, and then each system presented will be given as an extension of this initial system.  
The  initial inference rules, which just model
propositional intuitionistic logic, are as follows:
\begin{figure}
  \begin{mdframed}
\begin{center}
  \small
  \begin{math}
    \begin{array}{c}
      \begin{array}{cccccccc}
        \infer[\text{Id}]{\Delta,A \vdash A}{
          \,
        }
        & \quad &
        \infer[\text{Cut}]{\Gamma,\Delta \vdash C}{
          \Gamma \vdash B
          &
          B,\Delta \vdash C
        }
        & \quad & 
        \infer[\perp_{\mathcal{L}}]{\Gamma,\perp \; \vdash A}{
          \,
        }\\\\
        \infer[\lor_{\mathcal{L}}]{\Gamma,A \lor B \vdash C}{
          \Gamma,A \vdash C
          &
          \Gamma,B \vdash C
        }
        & \quad &
        \infer[\lor_{\mathcal{R}_1}]{\Gamma \vdash A \lor B}{
          \Gamma \vdash A
        }
        & \quad &
        \infer[\lor_{\mathcal{R}_2}]{\Gamma \vdash A \lor B}{
          \Gamma \vdash B
        }\\\\
        \infer[\land_{\mathcal{L}_1}]{\Gamma,A \land B \vdash C}{
          \Gamma,A \vdash C
        }
        & \quad &
        \infer[\land_{\mathcal{L}_2}]{\Gamma,A \land B \vdash C}{
          \Gamma,B \vdash C
        }
        & \quad &
        \infer[\land_{\mathcal{R}}]{\Gamma \vdash A \land B}{
          \Gamma \vdash A
          &
          \Gamma \vdash B
        }\\\\
        
      \end{array}
      \\
      \begin{array}{cccccccc}
        \infer[\to_{\mathcal{L}}]{\Gamma,A \to B \vdash C}{
          \Gamma \vdash A
          &
          \Gamma,B \vdash C
        }
        & \quad &
        \infer[\to_{\mathcal{R}}]{\Gamma \vdash A \to B}{
          \Gamma, A \vdash B
        }
      \end{array}        
    \end{array}
  \end{math}
\end{center}
 \end{mdframed}
  \caption{Intuitionistic Propositional  Calculus}
  \label{}
\end{figure}

Sequents denoted $\Gamma \vdash C$ consist of a multiset of formulas,
(written as either $\Gamma$, $\Delta$, or a numbered version of either), and a formula $C$. The intuitive meaning is that the conjunction of the formulas in $\Gamma$ entails the formula $C$.

\subsection{The constructive system S4} 

We recall the sequent calculus formalization of system
CS4.  This formalization  was proposed by Bierman and de Paiva \cite{CS4} and initially  was called  IS4 (for Intuitionistic system S4). This system gives rise to a complete family of systems, which are alternative to Simpson's intuitionistic systems~\cite{simpson1994}.

Simpson proposed in his thesis ~\cite{simpson1994}
systems  of intuitonsitic modal logics, forming a framework starting from intuitionistic K. These  were also called
IK, IKT, IS4, IS5, etc, hence it made sense to call the family of systems originating with the S4 above, the family of constructive modal systems CS, and Bierman and de Paiva's system CS4.  This is for instance how new work, such as \cite{arisaka2015}, describes these systems.

The main difference between systems IS4 and CS4 concerns the binary
distribution of possibility $\Diamond$ over disjunction (in binary and nullary forms):

\begin{figure}
  \begin{mdframed}
  \begin{center}
      \begin{math}
        \begin{array}{ccc}
        \Diamond (A \lor B) &\to &\Diamond A \lor \Diamond B\\
        \Diamond \bot &\to &\bot
        \end{array}
       \end{math}
\end{center}
 \end{mdframed}
  \caption{Distributiviy rules}
  \label{distrib}
\end{figure}

The system CS4 does not satisfy these
traditional distributions that are basic to classical modal
logic. Some philosophical significance can be attached to this
distribution, or lack thereof, and we discuss it later. First we
define the system.

The rules in Figure~\ref{fig:CS4}, in addition to the initial set of
inference rules, define the sequent calculus for CS4.
\begin{figure}
  \begin{mdframed}
    \begin{center}
      \begin{math}
        \begin{array}{ccccc}              
          \infer[\Box_{\mathcal{L}}]{\Gamma, \Box A \vdash B}{
            \Gamma,A \vdash B
          }
          & \quad &
          \infer[\Box_{\mathcal{R}}]{\Box\Gamma,\Delta \vdash \Box A}{
            \Box \Gamma \vdash A
          }\\\\
          \infer[\Diamond_{\mathcal{L}}]{\Delta,\Box\Gamma,\Diamond A \vdash \Diamond B}{
            \Box\Gamma,A \vdash \Diamond B
          }
          & \quad &
          \infer[\Diamond_{\mathcal{R}}]{\Gamma \vdash \Diamond A}{
            \Gamma \vdash A
          }
        \end{array}        
      \end{math}
    \end{center}
  \end{mdframed}
  \caption{CS4 modal rules}
  \label{fig:CS4}
\end{figure}

In Figure 3, we write $\Box \Gamma$ for $\Box G_1, \Box G_2, \ldots, \Box G_k$ where $\Gamma$ is the set $G_1, G_2, \ldots,  G_k$.
Note that we do have right rules and left  rules for introducing the new modal operators $\Box$ (necessity) and $\Diamond$ (posssibility), but these rules are not as symmetric as the propositional ones. Most importantly, we have a local restriciton on the rule that introduces the $\Box$ operator: We can only introduce $\Box$ in the conclusion, if all the assumptions are already boxed. Also the rules for  the $\Diamond$ operator presuppose that you have already $\Box$ operators.

This system is indeed constructive, and  $\Box A$ is not logically equivalent to $\lnot \Diamond \lnot A$, and $\Diamond A$ is
not logically equivalent to $\lnot \Box \lnot A$.

This system has a nice proof theory, as far as modal logics are concerned. Bierman and de Paiva \cite{CS4} show that it has a Hilbert-style presentation,  a Natural Deduction presentation, as well as a sequent calculus one. The sequent calculus
satisfies cut-elimination, an old result from Ohnishi and Matsumoto
\cite{ohnishi1957}, as well as the subformula property.
The Natural Deduction formulation has a colourful history, described
in \cite{CS4}. One of its distinct features is that it was described
in Prawitz' seminal book in Natural Deduction \cite{prawitz1965},
hence it is  sometimes called Prawitz S4 modal logic.

Most interestingly the system has both Kripke and categorical
semantics, described respectively in \cite{alechinaetal} and
\cite{CS4}. Since we can prove a Curry-Howard correspondence for this
system, it has been used in several applications within programming
language theory \todo{Add references for these applications}.
%Examples include \cite{hmm}.


\subsection{The adjunction modal calculus}
A slightly more type-theoretic system can be produced for the modal logic CS4. This is not so well-known, but this system can  be given a
presentation in terms of a categorical adjunction, between two
cartesian closed categories, as we will describe in the next section.

This presentation is described  both in \cite{CS4} and in \cite{icalp1998}. In
\cite{CS4} this is called the multicontext formulation of CS4 and the
rules are given (page 17) in Figure~\ref{fig:ADJCS4}.
\begin{figure}
  \begin{mdframed}
    \begin{center}
      \begin{math}
        \begin{array}{ccccc}              
          \infer[\Box_{\mathcal{I}}]{\Gamma; \Delta \vdash \Box A}{
            \Gamma;\emptyset \vdash  A
          }
          & \quad &
          \infer[\Box_{\mathcal{E}}]{\Gamma;\Delta \vdash B}{\Gamma; \Delta \vdash \Box A \hspace{.1in}
            \Gamma, A;\Delta \vdash B
          }\\\\
          \infer[\Diamond_{\mathcal{I}}]{\Gamma;\Delta \vdash \Diamond A}{
            \Gamma;\Delta \vdash A
          }
          & \quad &
          \infer[\Diamond_{\mathcal{E}}]{\Gamma;\Delta \vdash \Diamond B}{
            \Gamma ;\Delta \vdash \Diamond A \hspace{.1in}\Gamma; A\vdash \Diamond B
          }
        \end{array}        
      \end{math}
    \end{center}
  \end{mdframed}
  \caption{The adjunction modal calculus}
  \label{fig:ADJCS4}
\end{figure}
Note that the rules are Natural Deduction rules, as it should be clear from the fact that they are introduction and elimination rules.

These rules have been shown by Benton \cite{benton1995} and Barber to correspond to an adjunction of the categories in the case of Linear Logic.  We now describe the simplifications that can be made when, instead of Linear Logic, we deal with construcitve modal logic. 