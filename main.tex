\documentclass{article}
\usepackage[utf8]{inputenc}

\title{An introduction to that which shall not be named}
\author{Harley Eades III\\Computer Science\\Augusta University\\\href{mailto:heades@augusta.edu}{heades@augusta.edu}}
\date{September 2016}

\usepackage{graphicx}
\usepackage{amssymb, amsthm, amsmath, stmaryrd}
\usepackage{mathpartir}
\usepackage{mdframed}           % For the boxes around the systems.
\usepackage{cmll}
\usepackage[barr]{xy}

%% This renames Barr's \to to \mto.  This allows us to use \to for imp
%% and \mto for a inline morphism.
\let\mto\to
\let\to\relax
\newcommand{\to}{\rightarrow}

\usepackage{wasysym}
\usepackage{proof}
\usepackage{enumerate}
\usepackage{todonotes}
\usepackage{hyperref}
\usepackage{graphicx}

\let\b\relax
\let\d\relax
\let\t\relax

\newcommand{\F}{\mathop{\textbf{F}}}
\renewcommand{\P}{\mathop{\textbf{P}}}
\newcommand{\G}{\mathop{\textbf{G}}}
\renewcommand{\H}{\mathop{\textbf{H}}}

\newcommand{\cat}[1]{\mathcal{#1}}
\newcommand{\pd}[0]{\times}
\newcommand{\ihom}[0]{\rightarrow}
\newcommand{\st}[2]{\mathsf{st}_{#1,#2}}
\newcommand{\id}[0]{\mathsf{id}}
\newcommand{\b}[1]{\mathsf{b}_{#1}}
\newcommand{\d}[1]{\mathsf{d}_{#1}}
\newcommand{\m}[1]{\mathsf{m}_{#1}}
\newcommand{\n}[1]{\mathsf{n}_{#1}}
\newcommand{\p}[1]{\mathsf{p}_{#1}}
\newcommand{\q}[1]{\mathsf{q}_{#1}}
\newcommand{\t}[0]{\mathsf{t}}
\newcommand{\limp}[0]{\multimap}
\newcommand{\Hom}[3]{\mathsf{Hom}_{#1}(#2,#3)}

\def\cW{\color{white}}

%% Begin Ott

%% End Ott
                                   
\newtheorem{theorem}{Theorem}
\newtheorem{lemma}[theorem]{Lemma}
\newtheorem{example}[theorem]{Example}
\newtheorem{fact}[theorem]{Fact}
\newtheorem{corollary}[theorem]{Corollary}
\newtheorem{definition}[theorem]{Definition}
\newtheorem{remark}[theorem]{Remark}
\newtheorem{proposition}[theorem]{Proposition}
\newtheorem{notn}[theorem]{Notation}
\newtheorem{observation}[theorem]{Observation}

\begin{document}

\maketitle

\section{A quick overview from programming}
\label{sec:a_quick_example_from_programming}
Consider a C program with signature \verb!int f(int)!.  Now describe
all possible computations this program can do.  This is a difficult
task, because this function could do a lot, like, prompt the user for
input, send packets across the network, modify global state, and much
more, but then eventually return an integer.

Now consider a purely functional programming language like Haskell
\cite{Haskell} and list all the possible computations the function
\verb!f :: Int -> Int! can do. The list is a lot smaller.  We know
without a doubt that this function must take an integer as an input,
and then do integer computations, and finally return an integer.  No
funny business went on inside this function.  Thus, reasoning about
pure programs is a lot easier.

However, a practical programmer might now be asking, ``How do we get
any real work done in a pure setting?''.  From stage left enters the
monad.  These allow for a programmer to annotate the return types of
functions to indicate which side effects the function will use.  For
example, say we wanted \verb!f! to use a global state of integers,
then its type would be \verb!f :: Int -> State [Int] Int! to indicate
that \verb!f! will take in an integer input, then during computation
will use a global state consisting of a list of integers, but then
eventually return an integer.  Thus, the return type of \verb!f!
  literally lists the side effects the function will use.  Now while
  reasoning about programs we know exactly which side effects to
  consider.

In full generality a monad is a type constructor \verb!m : * -> *!
where \verb!*! is the universe of types.  Given a type \verb!a! we
call the type \verb!m a! the type of computations returning values
of type \verb!a!.  Thus, a function \verb!f : a -> m b! is a function that
takes in values of type \verb!a!, and then returns a computation that
will eventually return a value of type \verb!b!.

Suppose we have \verb!f :: a -> m b! and \verb!g :: b -> m c!, and we
wish to apply \verb!g! to the value returned by \verb!f!. This sounds
perfectly reasonable, but ordinary composition \verb!g.f! does not
suffice, because the return type of \verb!f! is not identical to the
input type of \verb!g!.  Thus, we must come up with a new type of
composition.  To accomplish this in Haskell we first need a new
operator called \textit{bind} which is denoted by
\verb!>>= :: m b -> (b -> m c) -> m c!.  Then composition of \verb!f!
and \verb!g! can be defined by
\verb!\x -> ((f x) >>= g) :: a -> m c!.

So we have a composition for functions whose type has the shape
\verb!a -> m b!, but any self respecting composition has an identity.
This implies that we need some function \verb!id :: a -> m a!, such
that, $\verb!(\x -> (f x) >>= id)! = \verb!f!$ and
$\verb!(\x -> (id x) >>= g)! = \verb!g!$.  This identity is denoted by
\verb!return :: a -> m a! in Haskell, and has to be taken as
additional structure, because it cannot be defined in terms of bind.

Using \verb!bind! and \verb!return! in combination with products, sum
types, and higher-order functions a large number of monads can be
defined, but what are monads really?
% section a_quick_example_from_programming (end)

\section{What is a monad really?}
\label{sec:what_is_a_monad_really}

% section what_is_a_monad_really (end)

\section{A short history}
\label{sec:a_short_history}
\cite{Moggi:1988}
% section a_short_history (end)

\section{Jumping inside a monad}
\label{sec:jumping_inside_a_monad}
\subsection{The Kleisli category}
\label{subsec:the_kleisli_category}

% subsection the_kleisli_category (end)

\subsection{The Eilenberg-Moore category}
\label{subsec:the_eilenberg-moore_category}

% subsection the_eilenberg-moore_category (end)

\subsection{Why not an example from linear logic?}
\label{subsec:why_not_an_example_from_linear_logic}

% subsection why_not_an_example_from_linear_logic (end)
% section jumping_inside_a_monad (end)

\section{Monads are modular, right?}
\label{sec:monads_are_modular,_right?}

% section monads_are_modular,_right? (end)


\bibliographystyle{plain}
\bibliography{references}

\end{document}

